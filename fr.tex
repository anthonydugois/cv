\documentclass[12pt]{article}

\usepackage[a4paper,total={6.5in,9.5in}]{geometry}
\usepackage[english,main=french]{babel}
\usepackage[utf8]{inputenc}
\usepackage[T1]{fontenc}
\usepackage[autostyle]{csquotes}
\usepackage{microtype}
\usepackage{titlesec}
\usepackage{tabularx}
\usepackage{booktabs}
\usepackage{enumitem}
\usepackage{eqparbox}
\usepackage{etoolbox}
\usepackage[backend=biber,sorting=none,style=ieee]{biblatex}
\usepackage{charter}
\usepackage{hyperref} % must be last

%%% Main content.
\newcommand{\myname}{Anthony Dugois}
\newcommand{\mymail}{anthony.dugois@ens-lyon.fr}
\newcommand{\myphone}{06 37 21 84 22}
\newcommand{\myaffiliation}{Doctorant en Informatique\\
  Laboratoire de l'Informatique du Parallélisme\\
  École Normale Supérieure de Lyon}

%%% CV title.
\newcommand{\mytitle}{%
  \raggedright

  {\normalfont\bfseries\huge\myname}
  
  \vspace{10pt}

  \begin{minipage}[t]{0.65\textwidth}
    \myaffiliation
  \end{minipage}%
  \begin{minipage}[t]{0.35\textwidth}
    \flushright
    \href{mailto:\mymail}{\mymail} \\
    \myphone
  \end{minipage}
}

%%% Section titles.
\titleformat{\section}{\normalfont\bfseries\Large}{}{}{}{}
\titlespacing{\section}{0pt}{28pt plus 4pt minus 4pt}{8pt plus 2pt minus 2pt}

\titleformat{\subsection}{\normalfont\bfseries\normalsize}{}{}{}{}
\titlespacing{\subsection}{0pt}{18pt plus 4pt minus 4pt}{8pt plus 2pt minus 2pt}

%%% Custom description list.
\newcommand{\cvitemsep}{6pt}
\newcommand{\cvlabelsep}{16pt}

\newcounter{cvitems}
\AtBeginEnvironment{cvitems}{%
  \stepcounter{cvitems}
  \edef\itemid{\arabic{cvitems}}}
\newcommand\cvitemformat[2][l]{\eqmakebox[listlabel@\itemid][#1]{#2}}
\newlist{cvitems}{description}{1}
\setlist[cvitems]{%
    labelwidth=\eqboxwidth{listlabel@\itemid},
    labelsep=\cvlabelsep,
    leftmargin=!,
    itemsep=\cvitemsep,
    format=\normalfont\cvitemformat}

\newcommand{\cvitem}[2]{\item[#1] #2}

%%% No indent space.
\setlength\parindent{0em}

%%% Link styling.
\hypersetup{colorlinks=true,urlcolor=magenta}

%%% Bibliography.
\addbibresource{main.bib}

\begin{document}

\mytitle

\section*{Expérience professionnelle}

\begin{cvitems}
  \cvitem{2020--}{\textbf{Doctorant à l'ENS Lyon}, encadré par Loris Marchal et Louis-Claude Canon
  au sein du LIP (Laboratoire de l'Informatique du Parallélisme).  
  Ordonnancement pour les \emph{key-value stores}.}

  \cvitem{2020}{\textbf{Stagiaire à FEMTO-ST} (Besançon), encadré par Louis-Claude Canon et Loris
  Marchal.  
  Synthèse bibliographique : ordonnancement de requêtes dans les bases de données répliquées (6
  mois).}

  \cvitem{2019}{\textbf{Stagiaire à l'Univ.\ Catholique de Louvain}
  (Louvain-la-Neuve, Belgique), encadré par Etienne Rivière.  
  Simulation à évènements discrets d'un système de \emph{key-value store} (1 mois).}

  \cvitem{2019}{\textbf{Stagiaire à l'ENS Lyon}, encadré par Loris Marchal et Louis-Claude Canon.  
  Initiation à la recherche : ordonnancement de requêtes dans les bases de données répliquées (2
  mois).}

  % \cvitem{2018}{\textbf{Stagiaire chez Numerica} (Montbéliard).  
  % Développement mobile et Internet des Objets (3 mois).}

  % \cvitem{2017}{\textbf{Stagiaire chez Akufen} (Montréal, Canada).  
  % Développement web (3 mois).}
\end{cvitems}

\section*{Formation}

\begin{cvitems}
  \cvitem{\bfseries Doctorat}{Thèse de Doctorat en Informatique menée à l'École Normale Supérieure
  de Lyon, encadrée par Loris Marchal et Louis-Claude Canon depuis octobre 2020.}

  \cvitem{\bfseries Master}{Master Informatique à l'Univ.\ de Franche-Comté (Besançon).  
  Ingénierie système et logiciels.  
  Mention très bien (major de promotion).  
  2018--2020.}

  \cvitem{\bfseries Licence}{Licence Informatique à l'Univ.\ de Franche-Comté (Besançon), précédée
  d'un DUT à l'IUT de Belfort-Montbéliard.  
  2015--2018.}

  \cvitem{\bfseries CPGE}{Cycle préparatoire (mathématiques, biologie, physique, chimie).  
  2013-2015.}
\end{cvitems}

\section*{Compétences}

\begin{cvitems}
  \cvitem{Académiques}{Théorie de l'ordonnancement, algorithmes d'approximation, algorithmique et programmes parallèles, systèmes distribués, réseaux, logique.}

  \cvitem{Techniques}{C, C++, Python, R, Java, SQL, MPI.}

  \cvitem{Linguistiques}{Anglais, Français.}
\end{cvitems}

\section*{Enseignement}

Les Travaux Dirigés (TD) et Travaux Pratiques (TP) se font en parallèle des activités de recherche.  
Pour chaque module, les effectifs des groupes se situent entre 10 et 15 étudiants.  
Le public concerné est issu de l'École Normale Supérieure de Lyon (ENSL) et de l'Université de
Franche-Comté (UFC).

\begin{center}
  \footnotesize
  \begin{tabularx}{\textwidth}{rXllll}
    \toprule
    Année & Module & Public & Niveau & Type & Durée (hTD) \tabularnewline
    \midrule
    2022--2023 & Bases de la programmation & UFC & L1 & TD/TP & 52 \tabularnewline
    & Réseaux & UFC & M1 & TP & 12 \tabularnewline
    \midrule
    2021--2022 & Circuits Logiques \& Réseaux & ENSL & L3 & TD/TP & 32 \tabularnewline
    & Algorithmes Parallèles et Prog.\ Distribués & ENSL & M1 & TD/TP & 32 \tabularnewline
    \midrule
    2020--2021 & Architecture, Système et Réseaux & ENSL & L3 & TD/TP & 32 \tabularnewline
    & Projet Intégré & ENSL & M1 & Projet & 32 \tabularnewline
    \bottomrule
  \end{tabularx}
\end{center}

\section*{Publications}

Les auteurs sont listés par ordre alphabétique.

\nocite{*}

\newcommand{\showbib}[1]{%
  \begin{otherlanguage}{english}
    \printbibliography[heading=none,keyword={#1}]
  \end{otherlanguage}}

\subsection*{Conférences internationales}

\showbib{international proceedings}

\subsection*{Rapports de recherche}

\showbib{research report}

\subsection*{Articles soumis (revue en cours)}

\showbib{under review}

\section*{Présentations}

\subsection*{Conférences internationales}

\begin{itemize}
  \item \foreignlanguage{english}{Bounding the Flow Time in Online Scheduling under Structured
  Processing Sets}, 1 juin 2022, IPDPS 2022, visio-conférence (en).
  \item \foreignlanguage{english}{Taming Tail-Latency in Key-Value Stores: a Scheduling
  Perspective}, 2 septembre 2021, EuroPar 2021, visio-conférence (en).
\end{itemize}

\subsection*{Séminaires}

\begin{itemize}
  \item \foreignlanguage{english}{Bounding the Flow Time in Online Scheduling under Structured
  Processing Sets}, 25 novembre 2022, Groupe de Travail GOThA, Metz (fr).
  \item \foreignlanguage{english}{Bounding the Flow Time in Online Scheduling under Structured
  Processing Sets}, 30 août 2022, Journée des doctorants, Mésandans (fr).
  \item \foreignlanguage{english}{Bounding the Flow Time in Online Scheduling under Structured
  Processing Sets}, 17 mai 2022, Scheduling Workshop, Aussois (en).
  \item \foreignlanguage{english}{A Scheduling Framework for Distributed Key-Value Stores and
  its Application to Tail Latency Minimization}, 13 avril 2022, Groupe de Travail SCALE, Besançon (fr).
  \item \foreignlanguage{english}{Bounding the Flow Time in Online Scheduling under Structured
  Processing Sets}, 3 décembre 2021, Groupe de Travail SCALE, Lyon (fr).
\end{itemize}

% \section*{Activités collectives}
% 
% \subsection*{Relectures}
% 
% EuroPar 2021

\end{document}
