\documentclass[12pt]{article}

\usepackage[a4paper,total={6.5in,9.5in}]{geometry}
\usepackage[english,main=french]{babel}
\usepackage[utf8]{inputenc}
\usepackage[T1]{fontenc}
\usepackage[autostyle]{csquotes}
\usepackage{microtype}
\usepackage{titlesec}
\usepackage{tabularx}
\usepackage{booktabs}
\usepackage{enumitem}
\usepackage{eqparbox}
\usepackage{etoolbox}
\usepackage[backend=biber,sorting=none,style=ieee]{biblatex}
\usepackage{charter}
\usepackage{hyperref} % must be last

%%% Main content.
\newcommand{\myname}{Anthony \textsc{Dugois}}
\newcommand{\mymail}{\href{mailto:anthony.dugois@univ-fcomte.fr}{anthony.dugois@univ-fcomte.fr}}
\newcommand{\myphone}{06\,37\,21\,84\,22}
\newcommand{\mywebsite}{\href{https://anthonydugois.github.io/}{https://anthonydugois.github.io/}}

%%% Custom description list.
\newcommand{\cvitemsep}{6pt}
\newcommand{\cvlabelsep}{16pt}
\newcommand{\cvitemformat}[2][l]{\eqmakebox[listlabel@\itemid][#1]{#2}}
\newcommand{\cvitem}[2]{\item[#1] #2}

\newcounter{cvitems}
\AtBeginEnvironment{cvitems}{%
    \stepcounter{cvitems}
    \edef\itemid{\arabic{cvitems}}}

\newlist{cvitems}{description}{1}
\setlist[cvitems]{%
    labelwidth=\eqboxwidth{listlabel@\itemid},
    labelsep=\cvlabelsep,
    leftmargin=!,
    itemsep=\cvitemsep,
    format=\normalfont\cvitemformat}

%%% Link styling.
\hypersetup{colorlinks=true,urlcolor=magenta}

%%% Bibliography.
\addbibresource{main.bib}

\nocite{*}

\begin{document}

\begin{center}
    {\LARGE\bfseries\myname}

    \vspace{10pt}

    {Attaché Temporaire d'Enseignement et de Recherche} \\[1pt]
    {à l'Université de Franche-Comté}
\end{center}

\section{État civil}

\noindent
\renewcommand{\arraystretch}{1.4}
\begin{tabularx}{\linewidth}{@{}p{3cm}X@{}}
    Nom         & \textsc{Dugois} \tabularnewline
    Prénom      & Anthony \tabularnewline
    Nationalité & Français \tabularnewline
    Né le       & 20 janvier 1995 \tabularnewline
    Adresse     & UFR Sciences et techniques \newline
                  Bâtiment Métrologie, Bureau 416 C \newline
                  16 route de Gray \newline
                  25\,000 Besançon \tabularnewline
    Téléphone   & \myphone \tabularnewline
    Courriel    & \mymail \tabularnewline
    Site web    & \mywebsite \tabularnewline
\end{tabularx}

\section{Parcours universitaire}

\begin{cvitems}
    \cvitem{2020--2023}{\textbf{Doctorat en Informatique} mené à l'École Normale Supérieure de Lyon
    et encadré par Loris \textsc{Marchal} et Louis-Claude \textsc{Canon}.

    \renewcommand{\arraystretch}{1.4}
    \begin{tabularx}{\linewidth}{@{}lX@{}}
        Titre         & \emph{\foreignlanguage{english}{Scheduling in Distributed Storage Systems}} \newline
                        Ordonnancement dans les systèmes de stockage distribués \tabularnewline
        Spécialité    & Informatique \tabularnewline
        Laboratoire   & Laboratoire de l'Informatique du Parallélisme (LIP) \tabularnewline
        Établissement & École Normale Supérieure de Lyon \tabularnewline
        Soutenance    & 28 septembre 2023 \tabularnewline
        Directeur     & Loris \textsc{Marchal} (HDR), CNRS, ENS Lyon, France \tabularnewline
        Co-encadrant  & Louis-Claude \textsc{Canon}, Univ.\ de Franche-Comté, France \tabularnewline
        Rapporteurs   & Oliver \textsc{Sinnen}, Univ.\ of Auckland, Nouvelle-Zélande \newline
                        Safia \textsc{Kedad-Sidhoum}, CNAM, France \tabularnewline
        Examinateurs  & Denis \textsc{Trystram}, Grenoble INP, France \newline
                        Sara \textsc{Bouchenak}, INSA Lyon, France \newline
                        Emmanuel \textsc{Jeannot}, Inria, France \tabularnewline
    \end{tabularx}}

    \cvitem{2018--2020}{\textbf{Master Informatique} à l'Univ.\ de Franche-Comté (Besançon).  
    Ingénierie système et logiciels.  
    Mémoire de recherche : \emph{Ordonnancement pour les bases de données répliquées}, encadré par
    Loris \textsc{Marchal} et Louis-Claude \textsc{Canon}.  
    Mention très bien (major de promotion). \\[4pt]
    Enseignements : compilation et interprétation, génie logiciel, théorie des graphes et
    combinatoire, réseaux avancés, architecture logicielle, intelligence artificielle,
    programmation multi-coeur, preuve de programmes, calculabilité, test fonctionnel, vérification
    automatique de modèles, initiation à la recherche.}

    \cvitem{2015--2018}{\textbf{Licence Informatique} à l'Univ.\ de Franche-Comté (Besançon),
    précédée d'un DUT à l'IUT de Belfort-Montbéliard. \\[4pt]
    Enseignements : outils pour la programmation, programmation fonctionnelle, sécurité, système,
    réseaux, théorie des langages, analyse syntaxique, programmation web, gestion de projet.}

    \cvitem{2013-2015}{\textbf{Cycle préparatoire} au Lycée Victor Hugo (Besançon). \\[4pt]
    Enseignements : mathématiques, biologie, géologie, physique, chimie.}
\end{cvitems}

\section{Expérience professionnelle}

\begin{cvitems}
    \cvitem{2023--2024}{\textbf{Attaché Temporaire d'Enseignement et de Recherche} à temps plein à
    l'Université de Franche-Comté dans l'équipe DEODIS du Département d'Informatique des Systèmes
    Complexes au sein du laboratoire FEMTO-ST (CDD, 1 an). \\[4pt]
    Service d'enseignement de 192 heqTD (L1, L2, L3, M1, M2). Encadrement de projets et de stages.
    Recherche : ordonnancement dans les systèmes de stockage distribués, ordonnancement
    \emph{online} avec modèles de prédiction.}

    \cvitem{2020--2023}{\textbf{Doctorant} à l'École Normale Supérieure de Lyon, encadré par Loris
    \textsc{Marchal} et Louis-Claude \textsc{Canon} au sein du LIP (bourse CORDI-S Inria, contrat
    doctoral, 3 ans). \\[4pt]
    3 services d'enseignement de 64 heqTD/an (L1, L3, M1). Suivi de projet et jury de stage.
    Recherche : ordonnancement dans les systèmes de stockage distribués.}

    \cvitem{2020}{\textbf{Stagiaire} au laboratoire FEMTO-ST (Besançon), encadré par Louis-Claude
    \textsc{Canon} et Loris \textsc{Marchal} (6 mois).  
    Synthèse bibliographique : ordonnancement de requêtes dans les bases de données répliquées.}

    \cvitem{2019}{\textbf{Stagiaire} à UCLouvain (Louvain-la-Neuve, Belgique), encadré par Etienne
    \textsc{Rivière} (1 mois).  
    Simulation à évènements discrets d'un \emph{key-value store}.}

    \cvitem{2019}{\textbf{Stagiaire} à l'École Normale Supérieure de Lyon, encadré par Loris
    \textsc{Marchal} et Louis-Claude \textsc{Canon} (2 mois).  
    Initiation à la recherche : ordonnancement de requêtes dans les bases de données répliquées.}
\end{cvitems}

\section{Compétences}

\begin{cvitems}
    \cvitem{\bfseries Académiques}{Systèmes distribués, optimisation, algorithmique, système,
    réseaux, bases de données, génie logiciel, compilation, logique, probabilités, statistiques.}

    \cvitem{\bfseries Techniques}{Prog.\ multi-paradigme (Python), prog.\ impérative (C), prog.\
    objet (Java, C++), prog.\ fonctionnelle (Scheme), prog.\ logique (Prolog), prog.\ web (PHP, SQL,
    JavaScript), scripts (Bash), science des données (R, NumPy, SciPy, Pandas, Scikit Learn,
    Jupyter), intégration \& déploiement continu (GitLab CI/CD, Docker, Ansible), solveurs linéaires
    (Gurobi).}

    \cvitem{\bfseries Linguistiques}{Français (langue maternelle), Anglais (lu, écrit, parlé).}
\end{cvitems}

\section{Activités d'enseignement}

J'ai eu l'occasion d'enseigner lors de mes 3 ans de doctorat de 2020 à 2023, sous le statut de
vacataire.  
Les publics concernés étaient des étudiants en \textbf{Master Informatique Fondamentale} (IF) de
l'École Normale Supérieure de Lyon (ENSL) et des étudiants en \textbf{Master Ingénierie Système et
Logiciels} (ISL) de l'Université de Franche-Comté (UFC).  
Ces enseignements ont principalement pris la forme de Travaux Dirigés (TD) et de Travaux Pratiques
(TP), à hauteur de 64 heqTD par an.  
J'enseigne désormais à l'UFC dans le cadre d'un ATER à hauteur de 192 heqTD par an, pour tous les
niveaux de la L1 au M2 inclus.

En plus des heures effectives, ma participation s'est également traduite par l'élaboration de
nouveaux sujets (projet, TP), l'amélioration de sujets existants, la correction d'examens et mon
investissement dans le fonctionnement des enseignements (réunions pédagogiques, jurys, surveillance,
etc.).

% Quelques exemples de supports et ressources que j'ai pu mettre en place :
% \begin{itemize}
%     \item Corrections de TD :
%     \href{https://anthonydugois.github.io/resources/appd-td5-solution.pdf}{appd-td5-solution.pdf},
%     \href{https://anthonydugois.github.io/resources/appd-td6-solution.pdf}{appd-td6-solution.pdf}
%     \item Sujet de TP :
%     \href{https://anthonydugois.github.io/resources/tf-tp3.pdf}{tf-tp3.pdf}
%     \item Sujets de projet :
%     \href{https://anthonydugois.github.io/resources/proj-sujet1.pdf}{proj-sujet1.pdf},
%     \href{https://anthonydugois.github.io/resources/proj-sujet2.pdf}{proj-sujet2.pdf}
% \end{itemize}

\subsection{Liste des enseignements}

Le tableau ci-dessous liste les enseignements que j'ai pu assurer de 2020 à 2024.  
Les volumes horaires sont donnés en heqTD.

\begin{center}
    \footnotesize
    \begin{tabularx}{\linewidth}{p{2cm}lp{3cm}lXll}
        \toprule
        \textbf{Année} & \textbf{Statut} & \textbf{Public} & \textbf{Niv.} & \textbf{Matière} & \textbf{Vol.} & \textbf{Nature} \tabularnewline
        \midrule
        2023--2024 & ATER & Master ISL (UFC) & M2 & Test Fonctionnel & 15 & TD/TP \tabularnewline
        & & & M1 & Compilation et Interprétation & 12 & TP \tabularnewline
        & & & M1 & Réseaux & 12 & TP \tabularnewline
        & & & L3 & Encadrement Projet & 8 & --- \tabularnewline
        & & & L3 & Encadrement Stage & 10 & --- \tabularnewline
        & & & L3 & Analyse Syntaxique & 24 & TP \tabularnewline
        & & & L3 & Développement Web Avancé & 24 & TP \tabularnewline
        & & & L2 & Langages du Web & 22 & TP \tabularnewline
        & & & L1 & Bases de Programmation & 56 & TD/TP \tabularnewline
        & & & L1 & Analyse et Traitement de Données & 10 & TP \tabularnewline
        \midrule
        2022--2023 & Vac. & Master ISL (UFC) & M1 & Réseaux & 12 & TP \tabularnewline
        & & & L1 & Bases de Programmation & 52 & TD/TP \tabularnewline
        \midrule
        2021--2022 & Vac. & Master IF (ENSL) & M1 & Algo.\ Parallèles et Prog.\ Distribués & 32 & TD/TP \tabularnewline
        & & & L3 & Circuits Logiques et Réseaux & 32 & TD/TP \tabularnewline
        \midrule
        2020--2021 & Vac. & Master IF (ENSL) & M1 & Projet & 30 & TP \tabularnewline
        & & & M1 & Stage & 4 & Jury \tabularnewline
        & & & L3 & Architecture, Système et Réseaux & 30 & TD/TP \tabularnewline
        \bottomrule
    \end{tabularx}
\end{center}

\section{Activités de recherche}

Mes recherches portent sur \textbf{l'optimisation et la prédictibilité des systèmes distribués} par
le biais de \textbf{l'ordonnancement des opérations} sur les ressources disponibles.  
Je m'intéresse particulièrement aux \textbf{systèmes de stockage distribués}, et j'utilise des
outils issus de \textbf{l'algorithmique} et de \textbf{la théorie de l'ordonnancement} (problèmes
d'optimisation, analyse de complexité, approximation, compétitivité, etc.) pour établir un cadre
d'étude formel.  
Mes travaux se sont déroulés dans le cadre de mon stage de fin de Master, puis de ma thèse de
doctorat à l'ENS Lyon et à l'Université de Franche-Comté (sous la direction de Loris
\textsc{Marchal} et Louis-Claude \textsc{Canon}), ainsi qu'à l'UCLouvain en Belgique (dans le cadre
d'une collaboration avec Etienne \textsc{Rivière}).  
Je poursuis actuellement ces travaux dans le cadre d'un ATER à l'Université de Franche-Comté.

% \subsection{Résumé des travaux}

% Durant la thèse, nous avons proposé une nouvelle approche pour l'étude d'une classe bien précise de
% systèmes de stockage distribués, à savoir les bases de données de type clé/valeur (plus communément
% appelées \emph{key-value stores}).  
% Cette approche est basée en grande partie sur la théorie de l'ordonnancement.  
% Nous avons poursuivi trois objectifs principaux : (i) le développement de garanties théoriques
% générales sur les métriques de performance classiques, à savoir la latence et le débit, (ii) le
% développement d'outils formels et pratiques pour faciliter l'évaluation de différentes
% caractéristiques du système et (iii) le développement de principes permettant de guider les
% optimisations futures.

% Lors de ma première année de doctorat, nous avons proposé un modèle formel pour l'étude de problèmes
% d'ordonnancement dans les \emph{key-value stores} persistants, en particulier le problème de
% minimisation du temps de réponse maximum (\emph{max-flow}) d'un flux de requêtes.  
% Ceci nous a permis de dériver des résultats théoriques (optimalité et approximation) sur des
% variantes simplifiées, le problème initial étant \(\mathbf{NP}\)-difficile au sens fort.  
% Nous avons identifié la difficulté principale du problème de minimisation, à savoir la contrainte de
% localité des données, i.e., une requête ne peut être exécutée que par le sous-ensemble de serveurs
% stockant la clé recherchée.  
% De plus, en exploitant un résultat de la littérature, nous avons montré comment trouver une borne
% inférieure sur le temps de réponse maximum de n'importe quelle instance du problème, ce qui permet
% en pratique de comparer et d'évaluer des heuristiques d'ordonnancement au travers de simulations.  
% Nous avons ainsi mis en évidence l'efficacité d'une heuristique simple, consistant à placer chaque
% requête sur la machine qui la terminera au plus tôt.  
% Ces travaux ont fait l'objet d'un rapport de recherche~\cite{benmokhtar2021-rr} et d'une publication
% à la conférence Euro-Par 2021~\cite{benmokhtar2021}.  
% Une version étendue est en cours de relecture dans \emph{Journal of Scheduling}.

% Nous avons poursuivi l'exploration du modèle théorique en introduisant des restrictions sur les
% ensembles de machines capables d'exécuter une requête donnée.  
% Dans un \emph{key-value store}, la stratégie de réplication des données n'est pas arbitraire et
% définit une structure bien particulière au sein des ensembles de réplication.  
% Nous avons cherché à comprendre comment différents schémas de réplication peuvent impacter le temps
% de réponse et le débit du système.  
% En particulier, en considérant la variante \emph{online} du problème (où l'instance est découverte
% au fur et à mesure), nous nous sommes intéressés au ratio de compétitivité sur le temps de réponse
% maximum, i.e., la différence relative entre un algorithme d'ordonnancement \emph{online} et une
% stratégie optimale \emph{offline}, et nous avons dérivé des bornes théoriques fortes en fonction du
% schéma de réplication considéré.  
% Nous avons également conçu une méthode exacte permettant de calculer le débit maximum théoriquement
% atteignable en fonction de la stratégie de réplication et de la distribution des fréquences d'accès
% aux données.  
% Ces travaux ont fait l'objet d'un rapport de recherche~\cite{canon2022-rr} et d'une publication à la
% conférence IPDPS 2022~\cite{canon2022}.

% Par le biais d'une collaboration avec le Professeur Etienne Rivière à l'UCLouvain en Belgique, nous
% avons ensuite étudié le problème sous un angle plus expérimental en nous intéressant à
% l'ordonnancement pratique dans Apache Cassandra, un système de \emph{key-value store} largement
% utilisé en production dans l'industrie.  
% Nous avons notamment proposé Hector, un framework modulaire permettant de résoudre plusieurs
% difficultés liées à la conception et à l'évaluation de politiques d'ordonnancement dans Apache
% Cassandra.  
% Après avoir vérifié qu'Hector n'apportait pas de surcoût significatif par rapport au système
% initial, ceci nous a permis de proposer de nouvelles stratégies ayant un effet bénéfique sur le
% débit et le temps de réponse du système.  
% En particulier, l'exploitation effective du cache du système d'exploitation peut améliorer
% significativement le débit du système.  
% Nous avons aussi montré que le ré-ordonnancement local des requêtes sur chaque machine peut
% améliorer le temps de réponse global en cas d'hétérogénéité de la charge de travail.  
% Ces travaux ont fait l'objet d'une publication à la conférence ICPP 2023~\cite{canon2023b}.

\subsection{Publications scientifiques}

Les auteurs et autrices sont listé(e)s par ordre alphabétique, conformément aux conventions en
vigueur dans les communautés liées aux travaux concernés.  
Pour chaque publication, j'ai contribué de manière significative à la revue de littérature, à la
définition du problème, aux analyses théoriques, et à la rédaction, excepté dans \cite{canon2024b},
où ma contribution a été plus faible sur les aspects théoriques.  
De plus, je suis le principal développeur de chaque logiciel et le concepteur des expérimentations
associées, excepté dans \cite{canon2024b}, où j'ai amélioré et étendu le logiciel existant.

\newcommand{\showbib}[1]{%
    \begin{otherlanguage}{english}
        \printbibliography[heading=none,keyword={#1}]
    \end{otherlanguage}}

\subsubsection*{Revues internationales avec comité de lecture}

\showbib{journals}

\subsubsection*{Actes de congrès international avec comité de lecture}

\showbib{international proceedings}

\subsubsection*{Actes de congrès national avec comité de lecture}

\showbib{national proceedings}

\subsubsection*{Rapports de recherche}

\showbib{research report}

\subsubsection*{Articles en cours de relecture, \emph{preprints}, etc.}

\showbib{misc}

\subsection{Communications scientifiques}

En plus des travaux précédemment décrits, j'ai eu l'occasion de présenter mes travaux de recherche à
plusieurs reprises, lors de conférences internationales et nationales, ainsi que lors de séminaires
scientifiques ou groupes de travail.

\subsubsection*{Conférences internationales avec comité de lecture}

\begin{itemize}
    \item \foreignlanguage{english}{Hector: A Framework to Design and Evaluate Scheduling Strategies
    in Persistent Key-Value Stores}, 9 août 2023, ICPP 2023, Salt Lake City, Utah (en).
    \url{https://icpp23.sci.utah.edu/program.html}
    \item \foreignlanguage{english}{Bounding the Flow Time in Online Scheduling under Structured
    Processing Sets}, 1 juin 2022, IPDPS 2022, visio-conférence (en).
    \url{https://ssl.linklings.net/conferences/ipdps/ipdps2022_program/views/at_a_glance.html}
    \item \foreignlanguage{english}{Taming Tail-Latency in Key-Value Stores: a Scheduling
    Perspective}, 2 septembre 2021, Euro-Par 2021, visio-conférence (en).
    \url{https://2021.euro-par.org/program/conference/}
\end{itemize}

\subsubsection*{Conférences nationales avec comité de lecture}

\begin{itemize}
  \item \foreignlanguage{english}{Hector: A Framework to Design and Evaluate Scheduling Strategies
  in Persistent Key-Value Stores}, 5 juillet 2023, COMPAS 2023, Annecy, France (fr).
  \url{https://2023.compas-conference.fr/programme/}
\end{itemize}

\subsubsection*{Séminaires scientifiques}

\begin{itemize}
  \item 25 novembre 2022, Groupe de Travail GOThA, Metz, France (fr).
  \item 30 août 2022, Journée des doctorants, Mésandans, France (fr).
  \item 17 mai 2022, Scheduling Workshop, Aussois, France (en).
  \item 13 avril 2022, Groupe de Travail SCALE, Besançon, France (fr).
  \item 3 décembre 2021, Groupe de Travail SCALE, Lyon, France (fr).
\end{itemize}

\subsection{Diffusions logicielles}

Les logiciels développés lors de ma thèse de doctorat sont disponibles en ligne sous licence libre,
ainsi que les données qui ont permis l'obtention des résultats présentés dans chaque publication.

\begin{itemize}
    \item \textbf{Simulation de \emph{key-value store}}: implémentation d'un simulateur à évènements
    discrets et évaluation de politiques d'ordonnancement
    (\url{https://doi.org/10.6084/m9.figshare.21750605.v1}, licence libre). Ce logiciel a permis
    d'obtenir plusieurs résultats dans \cite{benmokhtar2021,benmokhtar2024}. J'en suis le
    principal développeur (environ 2\,200 lignes).
    \item \textbf{Estimateur de débit théorique}: implémentation de la méthode exacte décrite dans
    \cite{canon2022} permettant de calculer le débit maximum théoriquement atteignable par un
    système de \emph{key-value store} en fonction de la stratégie de réplication et de la fréquence
    d'accès aux données (\url{https://doi.org/10.6084/m9.figshare.19123139.v1}, licence libre). J'en
    suis le principal développeur (environ 1\,100 lignes).
    \item \textbf{Hector}: framework modulaire permettant de faciliter l'implémentation et
    l'évaluation de nouvelles politiques d'ordonnancement dans Apache Cassandra
    (\url{https://github.com/anthonydugois/apache-cassandra-se}, licence libre). Les benchmarks
    permettant d'obtenir les résultats décrits dans \cite{canon2023b} sont également disponibles
    (\url{https://github.com/anthonydugois/hector-benchmarking}, licence libre). J'en suis le
    principal développeur (environ 15\,000 lignes).
    \item \textbf{Simulation de requêtes \emph{multi-get}}: implémentation d'un simulateur de
    requêtes \emph{multi-get} (individuelles ou en flux) et évaluation de stratégies de
    partitionnement. Ce logiciel a permis d'obtenir plusieurs résultats dans \cite{canon2024}. J'en
    suis le principal développeur (environ 3\,900 lignes).
\end{itemize}

\subsection{Responsabilités collectives}

\begin{itemize}
    \item Relecture d'article pour la conférence internationale
    \emph{\foreignlanguage{english}{Euro-Par 2021: Parallel Processing. 27th International European
    Conference on Parallel and Distributed Computing}}.
    \item Session chair à la conférence nationale COMPAS 2023 (session Parallélisme III):
    \url{https://2023.compas-conference.fr/programme/}.
\end{itemize}

\end{document}
