\documentclass[12pt]{article}

\usepackage[a4paper,total={6.5in,9.5in}]{geometry}
\usepackage[english,main=french]{babel}
\usepackage[utf8]{inputenc}
\usepackage[T1]{fontenc}
\usepackage[autostyle]{csquotes}
\usepackage{microtype}
\usepackage{titlesec}
\usepackage{tabularx}
\usepackage{booktabs}
\usepackage{enumitem}
\usepackage{eqparbox}
\usepackage{etoolbox}
\usepackage[backend=biber,sorting=none,style=ieee]{biblatex}
\usepackage{charter}
\usepackage{hyperref} % must be last

%%% Main content.
\newcommand{\myname}{Anthony \textsc{Dugois}}
\newcommand{\mymail}{\href{mailto:anthony.dugois@univ-fcomte.fr}{anthony.dugois@univ-fcomte.fr}}
\newcommand{\myphone}{06\,37\,21\,84\,22}
\newcommand{\mywebsite}{\href{https://anthonydugois.github.io/}{https://anthonydugois.github.io/}}
% \newcommand{\myaffiliation}{Doctorant en Informatique\\
%   Laboratoire de l'Informatique du Parallélisme\\
%   École Normale Supérieure de Lyon}

%%% CV title.
% \newcommand{\mytitle}{%
%   \raggedright

%   {\normalfont\bfseries\huge\myname}
  
%   \vspace{10pt}

%   \begin{minipage}[t]{0.65\textwidth}
%     \myaffiliation
%   \end{minipage}%
%   \begin{minipage}[t]{0.35\textwidth}
%     \flushright
%     \href{mailto:\mymail}{\mymail} \\
%     \myphone
%   \end{minipage}
% }

%%% Section titles.
% \titleformat{\section}{\normalfont\bfseries\Large}{}{}{}{}
% \titlespacing{\section}{0pt}{28pt plus 4pt minus 4pt}{8pt plus 2pt minus 2pt}

% \titleformat{\subsection}{\normalfont\bfseries\normalsize}{}{}{}{}
% \titlespacing{\subsection}{0pt}{18pt plus 4pt minus 4pt}{8pt plus 2pt minus 2pt}

%%% Custom description list.
\newcommand{\cvitemsep}{6pt}
\newcommand{\cvlabelsep}{16pt}
\newcommand{\cvitemformat}[2][l]{\eqmakebox[listlabel@\itemid][#1]{#2}}
\newcommand{\cvitem}[2]{\item[#1] #2}

\newcounter{cvitems}
\AtBeginEnvironment{cvitems}{%
    \stepcounter{cvitems}
    \edef\itemid{\arabic{cvitems}}}

\newlist{cvitems}{description}{1}
\setlist[cvitems]{%
    labelwidth=\eqboxwidth{listlabel@\itemid},
    labelsep=\cvlabelsep,
    leftmargin=!,
    itemsep=\cvitemsep,
    format=\normalfont\cvitemformat}

%%% Link styling.
\hypersetup{colorlinks=true,urlcolor=magenta}

%%% Bibliography.
\addbibresource{main.bib}

\nocite{*}

\begin{document}

\begin{center}
    {\LARGE\bfseries\myname}

    \vspace{10pt}

    {Attaché Temporaire d'Enseignement et de Recherche} \\[1pt]
    {à l'Université de Franche-Comté}
\end{center}

\section{État civil}

\noindent
\renewcommand{\arraystretch}{1.4}
\begin{tabularx}{\linewidth}{@{}p{3cm}X@{}}
    Nom         & \textsc{Dugois} \tabularnewline
    Prénom      & Anthony \tabularnewline
    Nationalité & Français \tabularnewline
    Né le       & 20 janvier 1995 \tabularnewline
    Adresse     & UFR Sciences et techniques \newline
                  Bâtiment Métrologie, Bureau 416 C \newline
                  16 route de Gray \newline
                  25\,000 Besançon \tabularnewline
    Téléphone   & \myphone \tabularnewline
    Courriel    & \mymail \tabularnewline
    Site web    & \mywebsite \tabularnewline
\end{tabularx}

\section{Parcours universitaire}

\begin{cvitems}
    \cvitem{2020--2023}{\textbf{Doctorat en Informatique} mené à l'École Normale Supérieure de Lyon
    et encadré par Loris \textsc{Marchal} et Louis-Claude \textsc{Canon}.

    \renewcommand{\arraystretch}{1.4}
    \begin{tabularx}{\linewidth}{@{}lX@{}}
        Titre         & \emph{\foreignlanguage{english}{Scheduling in Distributed Storage Systems}} \newline
                        Ordonnancement dans les systèmes de stockage distribués \tabularnewline
        Spécialité    & Informatique \tabularnewline
        Laboratoire   & Laboratoire de l'Informatique du Parallélisme (LIP) \tabularnewline
        Établissement & École Normale Supérieure de Lyon \tabularnewline
        Soutenance    & 28 septembre 2023 \tabularnewline
        Directeur     & Loris \textsc{Marchal} (HDR), CNRS, ENS Lyon, France \tabularnewline
        Co-encadrant  & Louis-Claude \textsc{Canon}, Univ.\ de Franche-Comté, France \tabularnewline
        Rapporteurs   & Oliver \textsc{Sinnen}, Univ.\ of Auckland, Nouvelle-Zélande \newline
                        Safia \textsc{Kedad-Sidhoum}, CNAM, France \tabularnewline
        Examinateurs  & Denis \textsc{Trystram}, Grenoble INP, France \newline
                        Sara \textsc{Bouchenak}, INSA Lyon, France \newline
                        Emmanuel \textsc{Jeannot}, Inria, France \tabularnewline
    \end{tabularx}}

    \cvitem{2018--2020}{\textbf{Master Informatique} à l'Univ.\ de Franche-Comté (Besançon).  
    Ingénierie système et logiciels.  
    Mémoire de recherche : \emph{Ordonnancement pour les bases de données répliquées}.  
    Encadré par Loris \textsc{Marchal} et Louis-Claude \textsc{Canon}.  
    Mention très bien (major de promotion).}

    \cvitem{2015--2018}{\textbf{Licence Informatique} à l'Univ.\ de Franche-Comté (Besançon),
    précédée d'un DUT à l'IUT de Belfort-Montbéliard.}

    \cvitem{2013-2015}{\textbf{Cycle préparatoire} (mathématiques, biologie, physique, chimie).}
\end{cvitems}

\section{Expérience professionnelle}

\begin{cvitems}
    \cvitem{2023--2024}{\textbf{ATER} à temps plein à l'Université de Franche-Comté dans l'équipe
    DEODIS du Département d'Informatique des Systèmes Complexes au sein du laboratoire FEMTO-ST
    (CDD, 1 an).}

    \cvitem{2020--2023}{\textbf{Doctorant} à l'École Normale Supérieure de Lyon, encadré par Loris
    \textsc{Marchal} et Louis-Claude \textsc{Canon} au sein du LIP (bourse CORDI-S Inria, contrat
    doctoral, 3 ans).}

    \cvitem{2020}{\textbf{Stagiaire} au laboratoire FEMTO-ST (Besançon), encadré par Louis-Claude
    \textsc{Canon} et Loris \textsc{Marchal} (6 mois).  
    Synthèse bibliographique : ordonnancement de requêtes dans les bases de données répliquées.}

    \cvitem{2019}{\textbf{Stagiaire} à UCLouvain (Louvain-la-Neuve, Belgique), encadré par Etienne
    \textsc{Rivière} (1 mois).  
    Simulation à évènements discrets d'un \emph{key-value store}.}

    \cvitem{2019}{\textbf{Stagiaire} à l'École Normale Supérieure de Lyon, encadré par Loris
    \textsc{Marchal} et Louis-Claude \textsc{Canon} (2 mois).  
    Initiation à la recherche : ordonnancement de requêtes dans les bases de données répliquées.}
\end{cvitems}

\section{Activités d'enseignement}

J'ai eu l'occasion d'enseigner lors de mes 3 ans de doctorat de 2020 à 2023, sous le statut de
vacataire.  
Les publics concernés étaient des étudiants en \textbf{Master Informatique Fondamentale} (IF) de
l'École Normale Supérieure de Lyon (ENSL) et des étudiants en \textbf{Master Ingénierie Système et
Logiciels} (ISL) de l'Université de Franche-Comté (UFC).  
Ces enseignements ont principalement pris la forme de Travaux Dirigés (TD) et de Travaux Pratiques
(TP), à hauteur de 64 heqTD\footnote{heqTD = heure équivalent TD.} par an.  
J'enseigne désormais à l'UFC dans le cadre d'un ATER à hauteur de 192 heqTD par an, pour tous les
niveaux de la L1 au M2 inclus.

En plus des heures effectives, ma participation s'est également traduite par l'élaboration de
nouveaux sujets (projet, TP), l'amélioration de sujets existants, la correction d'examens et mon
investissement dans le fonctionnement des enseignements (réunions pédagogiques, jurys, surveillance,
etc.).  
Quelques exemples de supports et ressources que j'ai pu mettre en place :
\begin{itemize}
    \item Corrections de TD :
    \href{https://anthonydugois.github.io/resources/appd-td5-solution.pdf}{appd-td5-solution.pdf},
    \href{https://anthonydugois.github.io/resources/appd-td6-solution.pdf}{appd-td6-solution.pdf}
    \item Sujet de TP :
    \href{https://anthonydugois.github.io/resources/tf-tp3.pdf}{tf-tp3.pdf}
    \item Sujets de projet :
    \href{https://anthonydugois.github.io/resources/proj-sujet1.pdf}{proj-sujet1.pdf},
    \href{https://anthonydugois.github.io/resources/proj-sujet2.pdf}{proj-sujet2.pdf}
\end{itemize}

\subsection{Liste des enseignements}

Le tableau ci-dessous liste les enseignements que j'ai pu assurer de 2020 à maintenant.  
Les modules apparaissant en italique sont ceux que j'assurerai lors du second semestre de l'année
universitaire 2023--2024.  
Les volumes horaires sont donnés en heqTD.

\begin{center}
    \footnotesize
    \begin{tabularx}{\linewidth}{p{2cm}lp{3cm}lXll}
        \toprule
        Année & Statut & Public & Niv. & Matière & Vol. & Nature \tabularnewline
        \midrule
        2023--2024 & ATER & Master ISL (UFC) & M2 & Test Fonctionnel & 15 & TD/TP \tabularnewline
        & & & M1 & Compilation et Interprétation & 12 & TP \tabularnewline
        & & & M1 & Réseaux & 12 & TP \tabularnewline
        & & & L3 & Développement Web Avancé & 24 & TP \tabularnewline
        & & & L3 & Encadrement Projet & 8 & --- \tabularnewline
        & & & L1 & Bases de Programmation & 56 & TD/TP \tabularnewline
        \cmidrule{4-7}
        & & & L3 & \emph{Encadrement Stage} & 10 & --- \tabularnewline
        & & & L3 & \emph{Analyse Syntaxique} & 24 & TP \tabularnewline
        & & & L2 & \emph{Langages du Web} & 22 & TP \tabularnewline
        & & & L1 & \emph{Analyse et Traitement de Données} & 10 & TP \tabularnewline
        \midrule
        2022--2023 & Vac. & Master ISL (UFC) & M1 & Réseaux & 12 & TP \tabularnewline
        & & & L1 & Bases de Programmation & 52 & TD/TP \tabularnewline
        \midrule
        2021--2022 & Vac. & Master IF (ENSL) & M1 & Algo.\ Parallèles et Prog.\ Distribués & 32 & TD/TP \tabularnewline
        & & & L3 & Circuits Logiques et Réseaux & 32 & TD/TP \tabularnewline
        \midrule
        2020--2021 & Vac. & Master IF (ENSL) & M1 & Projet & 30 & TP \tabularnewline
        & & & M1 & Stage & 4 & Jury \tabularnewline
        & & & L3 & Architecture, Système et Réseaux & 30 & TD/TP \tabularnewline
        \bottomrule
    \end{tabularx}
\end{center}

\section{Activités de recherche}

\subsection{Résumé des travaux}

Jusqu'à présent, mes travaux de recherche ont porté sur \textbf{l'optimisation et la prédictibilité
des systèmes de stockage distribués}, en particulier les bases de données de type clé/valeur, plus
communément appelées \emph{key-value stores}.  
Durant la thèse, nous avons proposé une nouvelle approche pour l'étude de ces systèmes, basée en
grande partie sur la \textbf{théorie de l'ordonnancement}.  
Nous avons poursuivi trois objectifs principaux : (i) le développement de garanties théoriques
générales sur les métriques de performance classiques, à savoir la latence et le débit, (ii) le
développement d'outils formels et pratiques pour faciliter l'évaluation de différentes
caractéristiques du système et (iii) le développement de principes permettant de guider les
optimisations futures.

Lors de ma première année de doctorat, nous avons proposé un modèle formel pour l'étude de problèmes
d'ordonnancement dans les \emph{key-value stores} persistants, en particulier le problème de
minimisation du temps de réponse maximum (\emph{max-flow}) d'un flux de requêtes.  
Ceci nous a permis de dériver des résultats théoriques (optimalité et approximation) sur des
variantes simplifiées, le problème initial étant \(\mathbf{NP}\)-difficile au sens fort.  
Nous avons identifié la difficulté principale du problème de minimisation, à savoir la contrainte de
localité des données, i.e., une requête ne peut être exécutée que par le sous-ensemble de serveurs
stockant la clé recherchée.  
De plus, en exploitant un résultat de la littérature, nous avons montré comment trouver une borne
inférieure sur le temps de réponse maximum de n'importe quelle instance du problème, ce qui permet
en pratique de comparer et d'évaluer des heuristiques d'ordonnancement au travers de simulations.  
Nous avons ainsi mis en évidence l'efficacité d'une heuristique simple, consistant à placer chaque
requête sur la machine qui la terminera au plus tôt.  
Ces travaux ont fait l'objet d'un rapport de recherche~\cite{benmokhtar2021-rr} et d'une publication
à la conférence Euro-Par 2021~\cite{benmokhtar2021}.  
Une version étendue est en cours de relecture dans \emph{Journal of Scheduling}.

Nous avons poursuivi l'exploration du modèle théorique en introduisant des restrictions sur les
ensembles de machines capables d'exécuter une requête donnée.  
Dans un \emph{key-value store}, la stratégie de réplication des données n'est pas arbitraire et
définit une structure bien particulière au sein des ensembles de réplication.  
Nous avons cherché à comprendre comment différents schémas de réplication peuvent impacter le temps
de réponse et le débit du système.  
En particulier, en considérant la variante \emph{online} du problème (où l'instance est découverte
au fur et à mesure), nous nous sommes intéressés au ratio de compétitivité sur le temps de réponse
maximum, i.e., la différence relative entre un algorithme d'ordonnancement \emph{online} et une
stratégie optimale \emph{offline}, et nous avons dérivé des bornes théoriques fortes en fonction du
schéma de réplication considéré.  
Nous avons également conçu une méthode exacte permettant de calculer le débit maximum théoriquement
atteignable en fonction de la stratégie de réplication et de la distribution des fréquences d'accès
aux données.  
Ces travaux ont fait l'objet d'un rapport de recherche~\cite{canon2022-rr} et d'une publication à la
conférence IPDPS 2022~\cite{canon2022}.

Par le biais d'une collaboration avec le Professeur Etienne Rivière à l'UCLouvain en Belgique, nous
avons ensuite étudié le problème sous un angle plus expérimental en nous intéressant à
l'ordonnancement pratique dans Apache Cassandra, un système de \emph{key-value store} largement
utilisé en production dans l'industrie.  
Nous avons notamment proposé Hector, un framework modulaire permettant de résoudre plusieurs
difficultés liées à la conception et à l'évaluation de politiques d'ordonnancement dans Apache
Cassandra.  
Après avoir vérifié qu'Hector n'apportait pas de surcoût significatif par rapport au système
initial, ceci nous a permis de proposer de nouvelles stratégies ayant un effet bénéfique sur le
débit et le temps de réponse du système.  
En particulier, l'exploitation effective du cache du système d'exploitation peut améliorer
significativement le débit du système.  
Nous avons aussi montré que le ré-ordonnancement local des requêtes sur chaque machine peut
améliorer le temps de réponse global en cas d'hétérogénéité de la charge de travail.  
Ces travaux ont fait l'objet d'une publication à la conférence ICPP 2023~\cite{canon2023b}.

\subsection{Publications scientifiques}

Les auteurs et autrices sont listé(e)s par ordre alphabétique, conformément aux conventions en
vigueur dans les communautés liées aux travaux concernés.

\newcommand{\showbib}[1]{%
    \begin{otherlanguage}{english}
        \printbibliography[heading=none,keyword={#1}]
    \end{otherlanguage}}

\subsubsection*{Conférences internationales avec comité de lecture}

\showbib{international proceedings}

\subsubsection*{Rapports de recherche}

\showbib{research report}

% \subsubsection*{Articles soumis (revue en cours)}

% \showbib{under review}

\subsection{Communications scientifiques}

\subsubsection*{Conférences internationales avec comité de lecture}

\begin{itemize}
    \item \foreignlanguage{english}{Hector: A Framework to Design and Evaluate Scheduling Strategies
    in Persistent Key-Value Stores}, 9 août 2023, ICPP 2023, Salt Lake City, Utah (en).
    \item \foreignlanguage{english}{Bounding the Flow Time in Online Scheduling under Structured
    Processing Sets}, 1 juin 2022, IPDPS 2022, visio-conférence (en).
    \item \foreignlanguage{english}{Taming Tail-Latency in Key-Value Stores: a Scheduling
    Perspective}, 2 septembre 2021, Euro-Par 2021, visio-conférence (en).
\end{itemize}

\subsubsection*{Conférences nationales avec comité de lecture}

\begin{itemize}
  \item \foreignlanguage{english}{Hector: A Framework to Design and Evaluate Scheduling Strategies
  in Persistent Key-Value Stores}, 5 juillet 2023, COMPAS 2023, Annecy, France (fr).
\end{itemize}

\subsubsection*{Séminaires scientifiques}

\begin{itemize}
  \item \foreignlanguage{english}{Bounding the Flow Time in Online Scheduling under Structured
  Processing Sets}, 25 novembre 2022, Groupe de Travail GOThA, Metz, France (fr).
  \item \foreignlanguage{english}{Bounding the Flow Time in Online Scheduling under Structured
  Processing Sets}, 30 août 2022, Journée des doctorants, Mésandans, France (fr).
  \item \foreignlanguage{english}{Bounding the Flow Time in Online Scheduling under Structured
  Processing Sets}, 17 mai 2022, Scheduling Workshop, Aussois, France (en).
  \item \foreignlanguage{english}{A Scheduling Framework for Distributed Key-Value Stores and its
  Application to Tail Latency Minimization}, 13 avril 2022, Groupe de Travail SCALE, Besançon,
  France (fr).
  \item \foreignlanguage{english}{Bounding the Flow Time in Online Scheduling under Structured
  Processing Sets}, 3 décembre 2021, Groupe de Travail SCALE, Lyon, France (fr).
\end{itemize}

\subsection{Diffusions logicielles}

\begin{itemize}
    \item \textbf{Simulation de \emph{key-value store}}: implémentation d'un simulateur à évènements
    discrets et évaluation de politiques d'ordonnancement
    (\url{https://doi.org/10.6084/m9.figshare.21750605.v1}, licence libre). Ce logiciel a permis
    d'obtenir plusieurs résultats dans \cite{benmokhtar2021}. J'en suis le
    principal développeur.
    \item \textbf{Estimateur de débit théorique}: implémentation de la méthode exacte décrite dans
    \cite{canon2022} permettant de calculer le débit maximum théoriquement atteignable par un
    système de \emph{key-value store} en fonction de la stratégie de réplication et de la fréquence
    d'accès aux données (\url{https://doi.org/10.6084/m9.figshare.19123139.v1}, licence libre). J'en
    suis le principal développeur.
    \item \textbf{Hector}: framework modulaire permettant de faciliter l'implémentation et
    l'évaluation de nouvelles politiques d'ordonnancement dans Apache Cassandra
    (\url{https://github.com/anthonydugois/apache-cassandra-se}, licence libre). Les benchmarks
    permettant d'obtenir les résultats décrits dans \cite{canon2023b} sont également disponibles
    (\url{https://github.com/anthonydugois/hector-benchmarking}, licence libre). J'en suis le
    principal développeur.
\end{itemize}

\subsection{Responsabilités collectives}

\begin{itemize}
    \item Relecture d'article pour la conférence internationale
    \emph{\foreignlanguage{english}{Euro-Par 2021: Parallel Processing. 27th International European
    Conference on Parallel and Distributed Computing}}.
    \item Session chair à la conférence nationale COMPAS 2023 (session Parallélisme III):
    \url{https://2023.compas-conference.fr/programme/}.
\end{itemize}

% \section*{Compétences}

% \begin{cvitems}
%   \cvitem{Académiques}{Théorie de l'ordonnancement, algorithmes d'approximation, algorithmique et programmes parallèles, systèmes distribués, réseaux, logique.}

%   \cvitem{Techniques}{C, C++, Python, R, Java, SQL, MPI.}

%   \cvitem{Linguistiques}{Anglais, Français.}
% \end{cvitems}

% \section*{Publications}

% Les auteurs sont listés par ordre alphabétique.

% \nocite{*}

% \newcommand{\showbib}[1]{%
%   \begin{otherlanguage}{english}
%     \printbibliography[heading=none,keyword={#1}]
%   \end{otherlanguage}}

% \subsection*{Conférences internationales}

% \showbib{international proceedings}

% \subsection*{Rapports de recherche}

% \showbib{research report}

% \subsection*{Articles soumis (revue en cours)}

% \showbib{under review}

% \section*{Présentations}

% \subsection*{Conférences internationales}

% \begin{itemize}
%   \item \foreignlanguage{english}{Bounding the Flow Time in Online Scheduling under Structured
%   Processing Sets}, 1 juin 2022, IPDPS 2022, visio-conférence (en).
%   \item \foreignlanguage{english}{Taming Tail-Latency in Key-Value Stores: a Scheduling
%   Perspective}, 2 septembre 2021, EuroPar 2021, visio-conférence (en).
% \end{itemize}

% \subsection*{Conférences nationales}

% \begin{itemize}
%   \item \foreignlanguage{english}{Hector: A Framework to Design and Evaluate Scheduling Strategies
%   in Persistent Key-Value Stores}, 5 juillet 2023, COMPAS 2023, Annecy (fr).
% \end{itemize}

% \subsection*{Séminaires}

% \begin{itemize}
%   \item \foreignlanguage{english}{Bounding the Flow Time in Online Scheduling under Structured
%   Processing Sets}, 25 novembre 2022, Groupe de Travail GOThA, Metz (fr).
%   \item \foreignlanguage{english}{Bounding the Flow Time in Online Scheduling under Structured
%   Processing Sets}, 30 août 2022, Journée des doctorants, Mésandans (fr).
%   \item \foreignlanguage{english}{Bounding the Flow Time in Online Scheduling under Structured
%   Processing Sets}, 17 mai 2022, Scheduling Workshop, Aussois (en).
%   \item \foreignlanguage{english}{A Scheduling Framework for Distributed Key-Value Stores and
%   its Application to Tail Latency Minimization}, 13 avril 2022, Groupe de Travail SCALE, Besançon (fr).
%   \item \foreignlanguage{english}{Bounding the Flow Time in Online Scheduling under Structured
%   Processing Sets}, 3 décembre 2021, Groupe de Travail SCALE, Lyon (fr).
% \end{itemize}

% \section*{Activités collectives}
% 
% \subsection*{Relectures}
% 
% EuroPar 2021

\end{document}
