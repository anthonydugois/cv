\documentclass[12pt]{article}

\usepackage[a4paper,total={6.5in,9.5in}]{geometry}
\usepackage[english,main=french]{babel}
\usepackage[utf8]{inputenc}
\usepackage[T1]{fontenc}
\usepackage[autostyle]{csquotes}
\usepackage{microtype}
\usepackage{titlesec}
\usepackage{tabularx}
\usepackage{booktabs}
\usepackage{enumitem}
\usepackage{eqparbox}
\usepackage{etoolbox}
\usepackage[backend=biber,sorting=none,style=ieee]{biblatex}
\usepackage{charter}
\usepackage{hyperref} % must be last

%%% Main content.
\newcommand{\myname}{Anthony \textsc{Dugois}}
\newcommand{\mymail}{\href{mailto:anthony.dugois@univ-fcomte.fr}{anthony.dugois@univ-fcomte.fr}}
\newcommand{\myphone}{06\,37\,21\,84\,22}
\newcommand{\mywebsite}{\href{https://anthonydugois.github.io/}{https://anthonydugois.github.io/}}
% \newcommand{\myaffiliation}{Doctorant en Informatique\\
%   Laboratoire de l'Informatique du Parallélisme\\
%   École Normale Supérieure de Lyon}

%%% CV title.
% \newcommand{\mytitle}{%
%   \raggedright

%   {\normalfont\bfseries\huge\myname}
  
%   \vspace{10pt}

%   \begin{minipage}[t]{0.65\textwidth}
%     \myaffiliation
%   \end{minipage}%
%   \begin{minipage}[t]{0.35\textwidth}
%     \flushright
%     \href{mailto:\mymail}{\mymail} \\
%     \myphone
%   \end{minipage}
% }

%%% Section titles.
% \titleformat{\section}{\normalfont\bfseries\Large}{}{}{}{}
% \titlespacing{\section}{0pt}{28pt plus 4pt minus 4pt}{8pt plus 2pt minus 2pt}

% \titleformat{\subsection}{\normalfont\bfseries\normalsize}{}{}{}{}
% \titlespacing{\subsection}{0pt}{18pt plus 4pt minus 4pt}{8pt plus 2pt minus 2pt}

%%% Custom description list.
\newcommand{\cvitemsep}{6pt}
\newcommand{\cvlabelsep}{16pt}
\newcommand\cvitemformat[2][l]{\eqmakebox[listlabel@\itemid][#1]{#2}}
\newcommand{\cvitem}[2]{\item[#1] #2}

\newcounter{cvitems}
\AtBeginEnvironment{cvitems}{%
  \stepcounter{cvitems}
  \edef\itemid{\arabic{cvitems}}}

\newlist{cvitems}{description}{1}
\setlist[cvitems]{%
    labelwidth=\eqboxwidth{listlabel@\itemid},
    labelsep=\cvlabelsep,
    leftmargin=!,
    itemsep=\cvitemsep,
    format=\normalfont\cvitemformat}

%%% Link styling.
\hypersetup{colorlinks=true,urlcolor=magenta}

%%% Bibliography.
% \addbibresource{main.bib}

\begin{document}

\begin{center}
	{\LARGE\bfseries\myname}

	\vspace{10pt}

	{Attaché Temporaire d'Enseignement et de Recherche} \\[1pt]
	{Université de Franche-Comté}
\end{center}

\section{État civil}

\noindent
\renewcommand{\arraystretch}{1.4}
\begin{tabularx}{\linewidth}{@{}p{3cm}X@{}}
	Nom         & \textsc{Dugois} \tabularnewline
	Prénom      & Anthony \tabularnewline
	Nationalité & Français \tabularnewline
	Né le       & 20 janvier 1995 \tabularnewline
	Adresse     & UFR Sciences et techniques \newline
	              Bâtiment Métrologie, Bureau 416 C \newline
				  16 route de Gray \newline
				  25000 Besançon \tabularnewline
	Téléphone   & \myphone \tabularnewline
	Courriel    & \mymail \tabularnewline
	Site web    & \mywebsite \tabularnewline
\end{tabularx}

\section{Parcours universitaire}

\begin{cvitems}
	\cvitem{2020--2023}{\textbf{Doctorat en Informatique} mené à l'École Normale Supérieure de Lyon et
	encadré par Loris \textsc{Marchal} et Louis-Claude \textsc{Canon}.

	\renewcommand{\arraystretch}{1.4}
	\begin{tabularx}{\linewidth}{@{}lX@{}}
		Titre         & Ordonnancement dans les systèmes de stockage distribués \tabularnewline
		Spécialité    & Informatique \tabularnewline
		Laboratoire   & Laboratoire de l'Informatique du Parallélisme (LIP) \tabularnewline
		Établissement & École Normale Supérieure de Lyon \tabularnewline
		Soutenance    & 28 septembre 2023 \tabularnewline
		Directeur     & Loris \textsc{Marchal} (HDR), ENS Lyon, CNRS \tabularnewline
		Co-encadrant  & Louis-Claude \textsc{Canon}, Université de Franche-Comté \tabularnewline
		Rapporteurs   & Oliver \textsc{Sinnen}, University of Auckland (Nouvelle-Zélande) \newline
		                Safia \textsc{Kedad-Sidhoum}, CNAM \tabularnewline
		Examinateurs  & Denis \textsc{Trystram}, Grenoble INP \newline
		                Sara \textsc{Bouchenak}, INSA Lyon \newline
		                Emmanuel \textsc{Jeannot}, Inria \tabularnewline
	\end{tabularx}}

	\cvitem{2018--2020}{\textbf{Master Informatique} à l'Univ.\ de Franche-Comté (Besançon).  
	Ingénierie système et logiciels.  
	Mémoire de recherche : \emph{Ordonnancement pour les bases de données répliquées}.  
	Encadré par Loris \textsc{Marchal} et Louis-Claude \textsc{Canon}.  
	Mention très bien (major de promotion).}

	\cvitem{2015--2018}{\textbf{Licence Informatique} à l'Univ.\ de Franche-Comté (Besançon), précédée
	d'un DUT à l'IUT de Belfort-Montbéliard.}

	\cvitem{2013-2015}{\textbf{Cycle préparatoire} (mathématiques, biologie, physique, chimie).}
\end{cvitems}

\section{Expériences professionnelles}

\begin{cvitems}
	\cvitem{2023--2024}{\textbf{ATER} à temps plein à l'Université de Franche-Comté dans l'équipe
	DEODIS du Département Informatique et Systèmes Complexes au sein du laboratoire FEMTO-ST (CDD, 1
	an).}

	\cvitem{2020--2023}{\textbf{Doctorant} à l'École Normale Supérieure de Lyon, encadré par Loris
	\textsc{Marchal} et Louis-Claude \textsc{Canon} au sein du LIP (bourse CORDI-S Inria, contrat
	doctoral, 3 ans).}

	\cvitem{2020}{\textbf{Stagiaire} au laboratoire FEMTO-ST (Besançon), encadré par Louis-Claude
	\textsc{Canon} et Loris \textsc{Marchal} (6 mois).  
	Synthèse bibliographique : ordonnancement de requêtes dans les bases de données répliquées.}

	\cvitem{2019}{\textbf{Stagiaire} à UCLouvain (Louvain-la-Neuve, Belgique), encadré par Etienne
	\textsc{Rivière} (1 mois).  
	Simulation à évènements discrets d'un \emph{key-value store}.}

	\cvitem{2019}{\textbf{Stagiaire} à l'École Normale Supérieure de Lyon, encadré par Loris
	\textsc{Marchal} et Louis-Claude \textsc{Canon} (2 mois).  
	Initiation à la recherche : ordonnancement de requêtes dans les bases de données répliquées.}
\end{cvitems}

\section{Activités d'enseignement}

J'ai eu l'occasion d'enseigner lors de mes 3 ans de doctorat de 2020 à 2023, sous le statut de
vacataire.  
Les publics concernés étaient des étudiants de l'École Normale Supérieure de Lyon (ENSL) et des
étudiants de l'Université de Franche-Comté (UFC).  
Ces enseignements se sont principalement traduits par des Travaux Dirigés (TD) et des Travaux
Pratiques (TP), à hauteur de 64 heqTD par an.  
J'enseigne désormais à l'UFC dans le cadre d'un ATER à hauteur de 192 heqTD par an, pour tous les
niveaux de la L1 au M2 inclus.

Ma participation s'est traduite par l'élaboration de nouveaux sujets (projet, TP), l'amélioration de
sujets existants, la correction d'examens et mon investissement dans le fonctionnement des
enseignements (réunions pédagogiques, jurys, surveillance, etc.).

\begin{center}
	\footnotesize
	\begin{tabularx}{\linewidth}{p{2cm}p{3.5cm}lXll}
		\toprule
		Année & Public & Niv. & Matière & Vol. & Nature \tabularnewline
		\midrule
		2023--2024 (ATER) & Master Ingénierie Système et Logiciels (UFC) & M2 & Test Fonctionnel & 15 & TD/TP \tabularnewline
		& & M1 & Compilation et Interprétation & 12 & TP \tabularnewline
		& & M1 & Réseaux & 12 & TP \tabularnewline
		& & L3 & Web Avancé & 24 & TP \tabularnewline
		& & L3 & Analyse Syntaxique & 24 & TP \tabularnewline
		& & L2 & Langages du Web & 22 & TP \tabularnewline
		& & L1 & Bases de la Programmation & 56 & TD/TP \tabularnewline
		\midrule
		2022--2023 (Vacataire) & Master Ingénierie Système et Logiciels (UFC) & M1 & Réseaux & 12 & TP \tabularnewline
		& & L1 & Bases de la Programmation & 52 & TD/TP \tabularnewline
		\midrule
		2021--2022 (Vacataire) & Master Informatique Fondamentale (ENSL) & M1 & Algo.\ Parallèles et Programmes Distribués & 32 & TD/TP \tabularnewline
		& & L3 & Circuits Logiques et Réseaux & 32 & TD/TP \tabularnewline
		\midrule
		2020--2021 (Vacataire) & Master Informatique Fondamentale (ENSL) & M1 & Projet & 30 & TP \tabularnewline
		& & M1 & Stage & 4 & Jury \tabularnewline
		& & L3 & Architecture, Système et Réseaux & 30 & TD/TP \tabularnewline
		\bottomrule
	\end{tabularx}
\end{center}

\section{Activités de recherche}

\subsection{Résumé des travaux}

\subsection{Publications scientifiques}

\subsection{Séminaires scientifiques}

\subsection{Diffusions logicielles}

\section{Responsabilités collectives}

% \section*{Compétences}

% \begin{cvitems}
%   \cvitem{Académiques}{Théorie de l'ordonnancement, algorithmes d'approximation, algorithmique et programmes parallèles, systèmes distribués, réseaux, logique.}

%   \cvitem{Techniques}{C, C++, Python, R, Java, SQL, MPI.}

%   \cvitem{Linguistiques}{Anglais, Français.}
% \end{cvitems}

% \section*{Enseignement}

% Les Travaux Dirigés (TD) et Travaux Pratiques (TP) se font en parallèle des activités de recherche.  
% Pour chaque module, les effectifs des groupes se situent entre 10 et 15 étudiants.  
% Le public concerné est issu de l'École Normale Supérieure de Lyon (ENSL) et de l'Université de
% Franche-Comté (UFC).

% \begin{center}
%   \footnotesize
%   \begin{tabularx}{\textwidth}{rXllll}
%     \toprule
%     Année & Module & Public & Niveau & Type & Durée (hTD) \tabularnewline
%     \midrule
%     2022--2023 & Bases de la programmation & UFC & L1 & TD/TP & 52 \tabularnewline
%     & Réseaux & UFC & M1 & TP & 12 \tabularnewline
%     \midrule
%     2021--2022 & Circuits Logiques \& Réseaux & ENSL & L3 & TD/TP & 32 \tabularnewline
%     & Algorithmes Parallèles et Prog.\ Distribués & ENSL & M1 & TD/TP & 32 \tabularnewline
%     \midrule
%     2020--2021 & Architecture, Système et Réseaux & ENSL & L3 & TD/TP & 32 \tabularnewline
%     & Projet Intégré & ENSL & M1 & Projet & 32 \tabularnewline
%     \bottomrule
%   \end{tabularx}
% \end{center}

% \section*{Publications}

% Les auteurs sont listés par ordre alphabétique.

% \nocite{*}

% \newcommand{\showbib}[1]{%
%   \begin{otherlanguage}{english}
%     \printbibliography[heading=none,keyword={#1}]
%   \end{otherlanguage}}

% \subsection*{Conférences internationales}

% \showbib{international proceedings}

% \subsection*{Rapports de recherche}

% \showbib{research report}

% \subsection*{Articles soumis (revue en cours)}

% \showbib{under review}

% \section*{Présentations}

% \subsection*{Conférences internationales}

% \begin{itemize}
%   \item \foreignlanguage{english}{Bounding the Flow Time in Online Scheduling under Structured
%   Processing Sets}, 1 juin 2022, IPDPS 2022, visio-conférence (en).
%   \item \foreignlanguage{english}{Taming Tail-Latency in Key-Value Stores: a Scheduling
%   Perspective}, 2 septembre 2021, EuroPar 2021, visio-conférence (en).
% \end{itemize}

% \subsection*{Conférences nationales}

% \begin{itemize}
%   \item \foreignlanguage{english}{Hector: A Framework to Design and Evaluate Scheduling Strategies
%   in Persistent Key-Value Stores}, 5 juillet 2023, COMPAS 2023, Annecy (fr).
% \end{itemize}

% \subsection*{Séminaires}

% \begin{itemize}
%   \item \foreignlanguage{english}{Bounding the Flow Time in Online Scheduling under Structured
%   Processing Sets}, 25 novembre 2022, Groupe de Travail GOThA, Metz (fr).
%   \item \foreignlanguage{english}{Bounding the Flow Time in Online Scheduling under Structured
%   Processing Sets}, 30 août 2022, Journée des doctorants, Mésandans (fr).
%   \item \foreignlanguage{english}{Bounding the Flow Time in Online Scheduling under Structured
%   Processing Sets}, 17 mai 2022, Scheduling Workshop, Aussois (en).
%   \item \foreignlanguage{english}{A Scheduling Framework for Distributed Key-Value Stores and
%   its Application to Tail Latency Minimization}, 13 avril 2022, Groupe de Travail SCALE, Besançon (fr).
%   \item \foreignlanguage{english}{Bounding the Flow Time in Online Scheduling under Structured
%   Processing Sets}, 3 décembre 2021, Groupe de Travail SCALE, Lyon (fr).
% \end{itemize}

% \section*{Activités collectives}
% 
% \subsection*{Relectures}
% 
% EuroPar 2021

\end{document}
