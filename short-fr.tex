\documentclass[12pt]{article}

\usepackage[a4paper,total={6.5in,9.5in}]{geometry}
\usepackage[english,main=french]{babel}
\usepackage[utf8]{inputenc}
\usepackage[T1]{fontenc}
\usepackage[autostyle]{csquotes}
\usepackage{microtype}
\usepackage{titlesec}
\usepackage{tabularx}
\usepackage{booktabs}
\usepackage{enumitem}
\usepackage{eqparbox}
\usepackage{etoolbox}
\usepackage[backend=biber,sorting=none,style=ieee]{biblatex}
\usepackage{charter}
\usepackage{hyperref} % must be last

%%% Main content.
\newcommand{\myname}{Anthony \textsc{Dugois}}
\newcommand{\mymail}{\href{mailto:anthony.dugois@univ-fcomte.fr}{anthony.dugois@univ-fcomte.fr}}
\newcommand{\myphone}{06\,37\,21\,84\,22}
\newcommand{\mywebsite}{\href{https://anthonydugois.github.io/}{https://anthonydugois.github.io/}}

%%% Custom description list.
\newcommand{\cvitemsep}{6pt}
\newcommand{\cvlabelsep}{16pt}
\newcommand{\cvitemformat}[2][l]{\eqmakebox[listlabel@\itemid][#1]{#2}}
\newcommand{\cvitem}[2]{\item[#1] #2}

\newcounter{cvitems}
\AtBeginEnvironment{cvitems}{%
    \stepcounter{cvitems}
    \edef\itemid{\arabic{cvitems}}}

\newlist{cvitems}{description}{1}
\setlist[cvitems]{%
    labelwidth=\eqboxwidth{listlabel@\itemid},
    labelsep=\cvlabelsep,
    leftmargin=!,
    itemsep=\cvitemsep,
    format=\normalfont\cvitemformat}

%%% Link styling.
\hypersetup{colorlinks=true,urlcolor=magenta}

%%% Bibliography.
\addbibresource{main.bib}

\nocite{*}

\begin{document}

\begin{center}
    {\LARGE\bfseries\myname}

    \vspace{10pt}

    {Attaché Temporaire d'Enseignement et de Recherche} \\[1pt]
    {à l'Université de Franche-Comté}
\end{center}

\section{État civil}

\noindent
\renewcommand{\arraystretch}{1.4}
\begin{tabularx}{\linewidth}{@{}p{3cm}X@{}}
    Nom         & \textsc{Dugois} \tabularnewline
    Prénom      & Anthony \tabularnewline
    Nationalité & Français \tabularnewline
    Né le       & 20 janvier 1995 \tabularnewline
    Adresse     & UFR Sciences et techniques \newline
                  Bâtiment Métrologie, Bureau 416 C \newline
                  16 route de Gray \newline
                  25\,000 Besançon \tabularnewline
    Téléphone   & \myphone \tabularnewline
    Courriel    & \mymail \tabularnewline
    Site web    & \mywebsite \tabularnewline
\end{tabularx}

\section{Parcours universitaire}

\begin{cvitems}
    \cvitem{2020--2023}{\textbf{Doctorat en Informatique} mené à l'École Normale Supérieure de Lyon
    et encadré par Loris \textsc{Marchal} et Louis-Claude \textsc{Canon}.  
	Titre : Ordonnancement dans les systèmes de stockage distribués.  
	Soutenu le 28 septembre 2023.}

    \cvitem{2018--2020}{\textbf{Master Informatique} à l'Univ.\ de Franche-Comté (Besançon).  
    Ingénierie système et logiciels.  
    Mémoire de recherche : \emph{Ordonnancement pour les bases de données répliquées}, encadré par
    Loris \textsc{Marchal} et Louis-Claude \textsc{Canon}.  
    Mention très bien (major de promotion).}

    \cvitem{2015--2018}{\textbf{Licence Informatique} à l'Univ.\ de Franche-Comté (Besançon),
    précédée d'un DUT à l'IUT de Belfort-Montbéliard.}

    \cvitem{2013-2015}{\textbf{Cycle préparatoire} au Lycée Victor Hugo (Besançon).}
\end{cvitems}

\section{Expérience professionnelle}

\begin{cvitems}
    \cvitem{2023--2024}{\textbf{Attaché Temporaire d'Enseignement et de Recherche} à temps plein à
    l'Université de Franche-Comté dans l'équipe DEODIS du Département d'Informatique des Systèmes
    Complexes au sein du laboratoire FEMTO-ST (CDD, 1 an).}

    \cvitem{2020--2023}{\textbf{Doctorant} à l'École Normale Supérieure de Lyon, encadré par Loris
    \textsc{Marchal} et Louis-Claude \textsc{Canon} au sein du LIP (bourse CORDI-S Inria, contrat
    doctoral, 3 ans).}

    \cvitem{2020}{\textbf{Stagiaire} au laboratoire FEMTO-ST (Besançon), encadré par Louis-Claude
    \textsc{Canon} et Loris \textsc{Marchal} (6 mois).}

    \cvitem{2019}{\textbf{Stagiaire} à UCLouvain (Louvain-la-Neuve, Belgique), encadré par Etienne
    \textsc{Rivière} (1 mois).}

    \cvitem{2019}{\textbf{Stagiaire} à l'École Normale Supérieure de Lyon, encadré par Loris
    \textsc{Marchal} et Louis-Claude \textsc{Canon} (2 mois).}
\end{cvitems}

\section{Activités d'enseignement}

J'ai eu l'occasion d'enseigner lors de mes 3 ans de doctorat de 2020 à 2023, sous le statut de
vacataire.  
Les publics concernés étaient des étudiants en \textbf{Master Informatique Fondamentale} (IF) de
l'École Normale Supérieure de Lyon (ENSL) et des étudiants en \textbf{Master Ingénierie Système et
Logiciels} (ISL) de l'Université de Franche-Comté (UFC).  
Ces enseignements ont principalement pris la forme de Travaux Dirigés (TD) et de Travaux Pratiques
(TP), à hauteur de 64 heqTD par an.  
J'enseigne désormais à l'UFC dans le cadre d'un ATER à hauteur de 192 heqTD par an, pour tous les
niveaux de la L1 au M2 inclus.  
Le tableau ci-dessous liste les enseignements que j'ai pu assurer de 2020 à 2024.  
Les volumes horaires sont donnés en heqTD.

\begin{center}
    \footnotesize
    \begin{tabularx}{\linewidth}{p{2cm}lp{3cm}lXll}
        \toprule
        \textbf{Année} & \textbf{Statut} & \textbf{Public} & \textbf{Niv.} & \textbf{Matière} & \textbf{Vol.} & \textbf{Nature} \tabularnewline
        \midrule
        2023--2024 & ATER & Master ISL (UFC) & M2 & Test Fonctionnel & 15 & TD/TP \tabularnewline
        & & & M1 & Compilation et Interprétation & 12 & TP \tabularnewline
        & & & M1 & Réseaux & 12 & TP \tabularnewline
        & & & L3 & Encadrement Projet & 8 & --- \tabularnewline
        & & & L3 & Encadrement Stage & 10 & --- \tabularnewline
        & & & L3 & Analyse Syntaxique & 24 & TP \tabularnewline
        & & & L3 & Développement Web Avancé & 24 & TP \tabularnewline
        & & & L2 & Langages du Web & 22 & TP \tabularnewline
        & & & L1 & Bases de Programmation & 56 & TD/TP \tabularnewline
        & & & L1 & Analyse et Traitement de Données & 10 & TP \tabularnewline
        \midrule
        2022--2023 & Vac. & Master ISL (UFC) & M1 & Réseaux & 12 & TP \tabularnewline
        & & & L1 & Bases de Programmation & 52 & TD/TP \tabularnewline
        \midrule
        2021--2022 & Vac. & Master IF (ENSL) & M1 & Algo.\ Parallèles et Prog.\ Distribués & 32 & TD/TP \tabularnewline
        & & & L3 & Circuits Logiques et Réseaux & 32 & TD/TP \tabularnewline
        \midrule
        2020--2021 & Vac. & Master IF (ENSL) & M1 & Projet & 30 & TP \tabularnewline
        & & & M1 & Stage & 4 & Jury \tabularnewline
        & & & L3 & Architecture, Système et Réseaux & 30 & TD/TP \tabularnewline
        \bottomrule
    \end{tabularx}
\end{center}

\section{Activités de recherche}

Mes recherches portent sur \textbf{l'optimisation et la prédictibilité des systèmes distribués} par
le biais de \textbf{l'ordonnancement des opérations} sur les ressources disponibles.  
Je m'intéresse particulièrement aux \textbf{systèmes de stockage distribués}, et j'utilise des
outils issus de \textbf{l'algorithmique} et de \textbf{la théorie de l'ordonnancement} (problèmes
d'optimisation, analyse de complexité, approximation, compétitivité, etc.) pour établir un cadre
d'étude formel.

\subsection{Publications scientifiques}

Les auteurs et autrices sont listé(e)s par ordre alphabétique, conformément aux conventions en
vigueur dans les communautés liées aux travaux concernés.

\newcommand{\showbib}[1]{%
    \begin{otherlanguage}{english}
        \printbibliography[heading=none,keyword={#1}]
    \end{otherlanguage}}

\subsubsection*{Revues internationales avec comité de lecture}

\showbib{journals}

\subsubsection*{Actes de congrès international avec comité de lecture}

\showbib{international proceedings}

\subsubsection*{Articles en cours de relecture, \emph{preprints}, etc.}

\showbib{misc}

\subsection{Communications scientifiques}

J'ai eu l'occasion de présenter mes travaux de recherche à plusieurs reprises, lors de conférences
internationales et nationales, ainsi que lors de séminaires scientifiques ou groupes de travail.

\subsubsection*{Conférences internationales avec comité de lecture}

\begin{itemize}
    \item \foreignlanguage{english}{Hector: A Framework to Design and Evaluate Scheduling Strategies
    in Persistent Key-Value Stores}, 9 août 2023, ICPP 2023, Salt Lake City, Utah (en).
    \url{https://icpp23.sci.utah.edu/program.html}
    \item \foreignlanguage{english}{Bounding the Flow Time in Online Scheduling under Structured
    Processing Sets}, 1 juin 2022, IPDPS 2022, visio-conférence (en).
    \url{https://ssl.linklings.net/conferences/ipdps/ipdps2022_program/views/at_a_glance.html}
    \item \foreignlanguage{english}{Taming Tail-Latency in Key-Value Stores: a Scheduling
    Perspective}, 2 septembre 2021, Euro-Par 2021, visio-conférence (en).
    \url{https://2021.euro-par.org/program/conference/}
\end{itemize}

\subsubsection*{Séminaires scientifiques}

\begin{itemize}
  \item 25 novembre 2022, Groupe de Travail GOThA, Metz, France (fr).
  \item 30 août 2022, Journée des doctorants, Mésandans, France (fr).
  \item 17 mai 2022, Scheduling Workshop, Aussois, France (en).
  \item 13 avril 2022, Groupe de Travail SCALE, Besançon, France (fr).
  \item 3 décembre 2021, Groupe de Travail SCALE, Lyon, France (fr).
\end{itemize}

\subsection{Diffusions logicielles}

\begin{itemize}
    \item \textbf{Simulation de \emph{key-value store}}: implémentation d'un simulateur à évènements
    discrets et évaluation de politiques d'ordonnancement
    (\url{https://doi.org/10.6084/m9.figshare.21750605.v1}, licence libre).
    \item \textbf{Estimateur de débit théorique}: implémentation d'une méthode exacte permettant de
    calculer le débit maximum théoriquement atteignable par un système de \emph{key-value store} en
    fonction de la stratégie de réplication et de la fréquence d'accès aux données
    (\url{https://doi.org/10.6084/m9.figshare.19123139.v1}, licence libre).
    \item \textbf{Hector}: framework modulaire permettant de faciliter l'implémentation et
    l'évaluation de nouvelles politiques d'ordonnancement dans Apache Cassandra
    (\url{https://github.com/anthonydugois/apache-cassandra-se}, licence libre).
    \item \textbf{Simulation de requêtes \emph{multi-get}}: implémentation d'un simulateur de
    requêtes \emph{multi-get} (individuelles ou en flux) et évaluation de stratégies de
    partitionnement.
\end{itemize}

\subsection{Responsabilités collectives}

\begin{itemize}
    \item Relecture d'article pour la conférence internationale
    \emph{\foreignlanguage{english}{Euro-Par 2021: Parallel Processing. 27th International European
    Conference on Parallel and Distributed Computing}}.
    \item Session chair à la conférence nationale COMPAS 2023 (session Parallélisme III):
    \url{https://2023.compas-conference.fr/programme/}.
\end{itemize}

\end{document}
