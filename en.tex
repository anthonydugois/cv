\documentclass[12pt]{article}

\usepackage[a4paper,total={6.5in,9.5in}]{geometry}
\usepackage[english]{babel}
\usepackage[utf8]{inputenc}
\usepackage[T1]{fontenc}
\usepackage[autostyle]{csquotes}
\usepackage{microtype}
\usepackage{titlesec}
\usepackage{tabularx}
\usepackage{booktabs}
\usepackage{enumitem}
\usepackage{eqparbox}
\usepackage{etoolbox}
\usepackage[backend=biber,sorting=none,style=ieee]{biblatex}
\usepackage{charter}
\usepackage{hyperref} % must be last

%%% Main content.
\newcommand{\myname}{Anthony Dugois}
\newcommand{\mymail}{anthony.dugois@ens-lyon.fr}
\newcommand{\myphone}{+33 6 37 21 84 22}
\newcommand{\myaffiliation}{PhD Candidate in Computer Science\\
  Laboratoire de l'Informatique du Parallélisme\\
  École Normale Supérieure de Lyon}

%%% CV title.
\newcommand{\mytitle}{%
  \raggedright

  {\normalfont\bfseries\huge\myname}
  
  \vspace{10pt}

  \begin{minipage}[t]{0.65\textwidth}
    \myaffiliation
  \end{minipage}%
  \begin{minipage}[t]{0.35\textwidth}
    \flushright
    \href{mailto:\mymail}{\mymail} \\
    \myphone
  \end{minipage}
}

%%% Section titles.
\titleformat{\section}{\normalfont\bfseries\Large}{}{}{}{}
\titlespacing{\section}{0pt}{28pt plus 4pt minus 4pt}{8pt plus 2pt minus 2pt}

\titleformat{\subsection}{\normalfont\bfseries\normalsize}{}{}{}{}
\titlespacing{\subsection}{0pt}{18pt plus 4pt minus 4pt}{8pt plus 2pt minus 2pt}

%%% Custom description list.
\newcommand{\cvitemsep}{6pt}
\newcommand{\cvlabelsep}{16pt}

\newcounter{cvitems}
\AtBeginEnvironment{cvitems}{%
  \stepcounter{cvitems}
  \edef\itemid{\arabic{cvitems}}}
\newcommand\cvitemformat[2][l]{\eqmakebox[listlabel@\itemid][#1]{#2}}
\newlist{cvitems}{description}{1}
\setlist[cvitems]{%
    labelwidth=\eqboxwidth{listlabel@\itemid},
    labelsep=\cvlabelsep,
    leftmargin=!,
    itemsep=\cvitemsep,
    format=\normalfont\cvitemformat}

\newcommand{\cvitem}[2]{\item[#1] #2}

%%% No indent space.
\setlength\parindent{0em}

%%% Link styling.
\hypersetup{colorlinks=true,urlcolor=magenta}

%%% Bibliography.
\addbibresource{main.bib}

\begin{document}

\mytitle

\section*{Experience}

\begin{cvitems}
  \cvitem{2020--}{\textbf{PhD Candidate at ENS Lyon}, supervised by Loris Marchal and Louis-Claude
  Canon in the LIP (Laboratoire de l'Informatique du Parallélisme).  
  Scheduling in distributed key-value stores.}

  \cvitem{2020}{\textbf{Intern at FEMTO-ST} (Besançon), supervised by Louis-Claude Canon and Loris
  Marchal.  
  Bibliographic synthesis: request scheduling in distributed databases (6 months).}

  \cvitem{2019}{\textbf{Intern at Univ.\ Catholique de Louvain}
  (Louvain-la-Neuve, Belgique), sueprvised by Etienne Rivière.  
  Discrete-event simulation of key-value store systems (1 month).}

  \cvitem{2019}{\textbf{Intern at ENS Lyon}, supervised by Loris Marchal and Louis-Claude Canon.  
  Research initiation: request scheduling in replicated databases (2 months).}

  % \cvitem{2018}{\textbf{Stagiaire chez Numerica} (Montbéliard).  
  % Développement mobile et Internet des Objets (3 mois).}

  % \cvitem{2017}{\textbf{Stagiaire chez Akufen} (Montréal, Canada).  
  % Développement web (3 mois).}
\end{cvitems}

\section*{Education}

\begin{cvitems}
  \cvitem{\bfseries PhD}{PhD thesis in Computer Science at École Normale Supérieure de Lyon,
  supervised by Loris Marchal and Louis-Claude Canon since october 2020.}

  \cvitem{\bfseries Master}{Master (equivalent to a Master's degree) in Computer Science at Univ.\
  de Franche-Comté (Besançon).  
  System \& Software Engineering.  
  2018--2020.}

  \cvitem{\bfseries Licence}{Licence (equivalent to a Bachelor's degree) in Computer Science at
  Univ.\ de Franche-Comté (Besançon).  
  2015--2018.}
\end{cvitems}

\section*{Skills}

\begin{cvitems}
  \cvitem{Academic}{Distributed Systems, Parallel Algorithms, Scheduling Theory, Networks, Logic.}

  \cvitem{Technical}{C, Python, R, Java, SQL, MPI, XML, JavaScript, HTML/CSS.}

  \cvitem{Language}{English, French.}
\end{cvitems}

\section*{Teaching}

Tutorials (TD) and Practical Work (TP) are made in parallel to research activities.  
For each module, groupe sizes range from 10 to 15 students.  
The target audience comes from École Normale Supérieure de Lyon (ENSL) and Université de
Franche-Comté (UFC).

\begin{center}
  \footnotesize
  \begin{tabularx}{\textwidth}{rXllll}
    \toprule
    Year & Module & Audience & Level & Type & Duration (hTD) \tabularnewline
    \midrule
    2022--2023 & Basics of Computer Programming & UFC & L1 & TD/TP & 52 \tabularnewline
    & Networks & UFC & M1 & TP & 12 \tabularnewline
    \midrule
    2021--2022 & Logic Circuits \& Networks & ENSL & L3 & TD/TP & 32 \tabularnewline
    & Parallel Algorithms \& Distributed Programs & ENSL & M1 & TD/TP & 32 \tabularnewline
    \midrule
    2020--2021 & Architecture, System and Networks & ENSL & L3 & TD/TP & 32 \tabularnewline
    & Programming Project & ENSL & M1 & Projet & 32 \tabularnewline
    \bottomrule
  \end{tabularx}
\end{center}

\section*{Research Publications}

Authors are sorted in alphabetical order.

\nocite{*}

\newcommand{\showbib}[1]{%
  \begin{otherlanguage}{english}
    \printbibliography[heading=none,keyword={#1}]
  \end{otherlanguage}}

\subsection*{International Conference Proceedings}

\showbib{international proceedings}

\subsection*{Research Reports}

\showbib{research report}

% \subsection*{Articles soumis (revue en cours)}

% \showbib{under review}

\section*{Research Presentations}

\subsection*{International Conferences}

\begin{itemize}
  \item Bounding the Flow Time in Online Scheduling under Structured Processing Sets, 2022, june 1,
  IPDPS 2022, videoconference (en).
  \item Taming Tail-Latency in Key-Value Stores: a Scheduling Perspective, 2021, september 2,
  EuroPar 2021, videoconference (en).
\end{itemize}

\subsection*{Workshops}

\begin{itemize}
  \item Bounding the Flow Time in Online Scheduling under Structured Processing Sets, 2022, november
  25, Groupe de Travail GOThA, Metz (fr).
  \item Bounding the Flow Time in Online Scheduling under Structured Processing Sets, 2022, august
  30, Journée des doctorants, Mésandans (fr).
  \item Bounding the Flow Time in Online Scheduling under Structured Processing Sets, 2022, may 17,
  Scheduling Workshop, Aussois (en).
  \item A Scheduling Framework for Distributed Key-Value Stores and Application to Tail Latency
  Minimization, 2022, april 13, Groupe de Travail SCALE, Besançon (fr).
  \item Bounding the Flow Time in Online Scheduling under Structured Processing Sets, 2021, december
  3, Groupe de Travail SCALE, Lyon (fr).
\end{itemize}

% \section*{Activités collectives}
% 
% \subsection*{Relectures}
% 
% EuroPar 2021

\end{document}
